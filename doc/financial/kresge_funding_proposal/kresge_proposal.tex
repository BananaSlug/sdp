 \documentclass[11pt]{article}
\usepackage[pdftex]{graphicx}
\usepackage{siunitx}
\usepackage{verbatim}
\usepackage{float}
\usepackage{lastpage}
\usepackage{fancyhdr}
\usepackage[margin=1in, paperwidth=8.5in, paperheight=11in]{geometry}
%\usepackage{indentfirst}

% Header and Footer
\pagestyle{fancy}
%\fancyhead{} % clear all header fields
%\rhead{\vspace*{1em}\headone{David Goodman}\\
%	$118$ Hainline Road $\bullet$ Aptos, CA 95003\\
%	Phone: 714.363.1280\\
%	E-Mail: dagoodma@gmail.com}
%\cfoot{}

\renewcommand{\headrulewidth}{0pt}
%\renewcommand{\footrulewidth}{0.4pt}
%\cfoot{}
%\rfoot{Goodman \thepage}

\begin{document}
\begin{center} 
{
    {\Large \textbf{Autonomous Lifeguard}}\\
  	{\large Student Project Funds Proposal}
}
\end{center}

\section*{Abstract}
In the United States, there have been a reported 99 people that have drowned in the past year alone, with a good number of them occurring while a lifeguard was on duty. In that year, there were 63,000 individual cases where Lifeguards rescued around 63,000 individuals at risk of drowning. Last year alone, around 99 people have drowned with a good number of them occurring while a lifeguard was on duty. We are proposing an autonomous surface vehicle to aid and assist drowning victims during the critical minutes before help can reach them.\\


This project We are building an autonomous surface vessel that will aid a lifeguard on the beach in order to save someone from drowning. Our project allows a lifeguard to quickly target a drowning person with a magnifying scope that will obtain their GPS coordinates using calibrated encoders, and then communicate them to an autonomous boat located in the water beyond the shore break. The boat will, using its own on-board GPS, navigate itself to the person and allow them to hang on and stay afloat until the lifeguard arrives. This project aims to keep beaches safer by reducing the risk of drowning. This is an example of how I am most inclined to work on projects like this one that can preserve or improve the quality of life for all people.

This project is composed of two systems, a command center and autonomous vehicle, that will wirelessly communicate with each other. The command center will consist of a GPS-based scope mounted on a lifeguard post that will allow the user to obtain a coordinate location of a drowning victim. Once prompted, the command center communicates this waypoint to the autonomous vehicle stationed in the water. The vehicle will then navigate to the designated waypoint using an onboard GPS unit. Upon reaching this destination, the ASV will intelligently traverse the area until the drowning victim has grabbed onto the vehicle. The ASV will support the victim and allow them to rest while the lifeguard makes their way to the victim.

\section*{Narrative}

\subsection*{Background}

The city of Los Angeles has employed the use of a device named EMILY (Emergency Integrated Lifesaving Lanyard) on their state beaches to save drowning victims. The device is deployed from the shore and remotely controlled by a human operator. Although it offers assistance, there exists multiple drawbacks of the system as a whole. The first is that it requires an operator, meaning that a lifeguard will be occupied during a rescue. The second is that the device must fight against the wave breaks in order to reach a victim. This implies delay when navigating to the victim. The third drawback is that the device is limited by the field of view and skill of the operator.\\

We propose a fully autonomous system that will navigate to the location of a drowning victim, offering assistance as a lifeguard is deployed from shore. This eliminates the need for a human operator, allowing the lifeguard to swim out to the victim as the device navigates to the victim. The device will be stationed within the water at a certain distance from the shore and beyond the wave breaks so that it may arrive at a drowning victims location swiftly.  

\subsection*{Objective}

Motivated by the number of drowning cases each year, The Autonomous Lifeguard Senior Design Group aims to develop an assistive, life-preserving, vessel with the capability of navigating to and locating drowning individuals in open water. This vessel, the Autonomous Lifeguard Assistant (AtLAs), will be stationed in the open water on a beach or in the lake. The AtLAs will be coupled with an onshore control tower, the Command Post Acquisition System (ComPAS), that transmits the GPS coordinates of drowning individuals that have been spotted by a lifeguard. While the lifeguard swims to the drowning individual, the AtLAs will use its triple motor propulsion system to arrive at the victims location at least three times as fast as a lifeguard would. Furthermore, the entire Autonomous Lifeguard System is designed to be a useful and affordable product for the public. The ultimate goal is to create a practical and superior aquatic life-saving system that requires minimally invasive operation, in this case, allowing the lifeguard to make their way to the victim in parallel with the machine.

\subsection*{Procedure}


In order to accomplish our objective of designing and implementing a device that can save someone from drowning, we propose a three phase approach: design, testing, and integration. In the design stage, which we are currently in, a majority of the research was accomplished to establish a solution to the task, and physical and mathematical models were created to allow for an enlightened analysis of the proposed solution. Next, the models established in the design phase are to be rigorously examined in the testing phase to acquire real data to prove the accuracy and robustness of the model. Lastly, the integration stage, in which the minimum specifications are to be addressed, involves assembling the final product by combining all of the refined modules into a physical implementation that meets the specifications. The three phases encompass the design life of the project. 

Since mid November, our team has made significant progress in completing the research and design phase for our project. We began by doing extensive background research related to the sensors and actuators, as well as the basic physical capabilities  required. This led to modeling and physical calculation related to the buoyancy, hydrodynamics, motors,  , such as buoyancy calculation. Using that research, we formulated various initial solutions, and then, with the input of trusted and qualified mentors, proceeded to amend, revise, and ultimately converse onto the most effective  solution.

The testing phase includes a exquisitely thorough 


integration is an iterative process that may lead us back into the testing phase depending on the success that 


\subsection*{Qualifications}

\subsection*{Benefits}

\end{document}
