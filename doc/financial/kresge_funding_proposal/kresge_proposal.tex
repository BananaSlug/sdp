 \documentclass[11pt]{article}
\usepackage[pdftex]{graphicx}
\usepackage{siunitx}
\usepackage{verbatim}
\usepackage{float}
\usepackage{lastpage}
\usepackage{fancyhdr}
\usepackage[margin=1in, paperwidth=8.5in, paperheight=11in]{geometry}
%\usepackage{indentfirst}

% Header and Footer
\pagestyle{fancy}
%\fancyhead{} % clear all header fields
%\rhead{\vspace*{1em}\headone{David Goodman}\\
%	$118$ Hainline Road $\bullet$ Aptos, CA 95003\\
%	Phone: 714.363.1280\\
%	E-Mail: dagoodma@gmail.com}
%\cfoot{}

\renewcommand{\headrulewidth}{0pt}
%\renewcommand{\footrulewidth}{0.4pt}
%\cfoot{}
%\rfoot{Goodman \thepage}

\begin{document}
\begin{center} 
{
    {\Large \textbf{Autonomous Lifeguard}}\\
  	{\large Student Project Funds Proposal}
}
\end{center}

\section*{Abstract}
In the United States, there have been a reported 99 people that have drowned in the past year alone, with a good number of them occurring while a lifeguard was on duty. In that year, there were 63,000 individual cases where Lifeguards rescued people at risk of drowning. We are proposing an autonomous surface vehicle to aid and assist these drowning individuals during the critical minutes before help can reach them.\\

Our project allows a lifeguard to quickly target a drowning person with a magnifying scope that will obtain their GPS coordinates and communicate them to a boat located in the water beyond the shore break. The boat will autonomously navigate to the person, allowing them to hang on and stay afloat until the lifeguard arrives. This project aims to keep beaches safer by reducing the risk of drowning. 


\section*{Narrative}

\subsection*{Background}
The city of Los Angeles has employed the use of a device named EMILY (Emergency Integrated Lifesaving Lanyard) on their state beaches to save drowning victims. The device is deployed from the shore and remotely controlled by a human operator. Although it offers a degree of assistance, there exist multiple drawbacks to the system as a whole. The first is that it requires an operator, meaning that a lifeguard will be occupied with controlling the device during a rescue. The second is that the device must fight against the waves in order to reach the victim. This implies an unpredictable delay when navigating to the victim. The third drawback is that the device is limited by the field of view and skill of the operator.\\

We propose a fully autonomous system that will navigate to the location of a drowning victim, offering assistance as a lifeguard is deployed from shore. This eliminates the need for a human operator, allowing the lifeguard to swim out to the victim as the device navigates to the victim. The device will be stationed within the water at a certain distance from the shore and beyond the wave breaks so that it may arrive at a drowning victim's location swiftly.

\subsection*{Objective}

Motivated by the number of drowning cases each year, The Autonomous Lifeguard Senior Design Group aims to develop an assistive, life-preserving, vessel with the capability of navigating to and locating drowning individuals in open water. This vessel, the Autonomous Lifeguard Assistant (AtLAs), will be stationed in the open water on a beach or in a lake. The AtLAs will be coupled with an onshore control tower, the Command Post Acquisition System (ComPAS), that transmits the GPS coordinates of drowning individuals that have been spotted by a lifeguard. While the lifeguard swims to the drowning individual, the AtLAs will use its triple motor propulsion system to arrive at the victim's location at least three times as fast as a lifeguard can. Furthermore, the entire Autonomous Lifeguard System is designed to be a useful and affordable product for the public. The ultimate goal is to create a practical and superior aquatic life-saving system.

\subsection*{Procedure}
This project is composed of two systems, a command center (ComPAS) and Autonomous Lifeguard Assistant (AtLAs), which will communicate with each other over a wireless protocol. The ComPAS consists of a GPS-equipped scope mounted on a lifeguard post. When the lifeguard sees someone who is drowning, he or she will spot them through the scope and press a button. At that instant, the coordinate location of the victim will be obtained--through triangulation algorithms and a global coordinate system provided by the GPS--and sent to the AtLAs in the water. The AtLAs will then begin to self navigate to the location using its own on-board GPS device as well as an array of sensors for detecting the person. Upon reaching this destination, the AtLAs will intelligently traverse the area until it finds the drowning victim. The AtLAs will support the victim and allow them to rest while the lifeguard makes their way out.\\

In order to accomplish our objective, we propose a three phase approach: design, integration, and testing. In the design stage, which we are currently in, a majority of the research was accomplished to establish a solution to the task and physical and mathematical models were created to allow for an enlightened analysis of the proposed solution. Next, the models established in the design phase are to be rigorously examined in the testing phase to acquire real data to prove the accuracy and robustness of the model. Lastly, the integration stage, in which the minimum specifications will be addressed, involves assembling the final product by combining all the refined modules into a physical implementation that meets the specifications. These three phases encompass the life of the project. \\

Since mid-November, our team has made significant progress in completing the research and design phase for our project. We began by doing extensive background research related to the sensors and actuators, as well as the basic physical capabilities required. This led to modeling and physical calculation related to the buoyancy, hydrodynamics, motors, power budget, sensor specifications, and the accuracy of our navigation system. Using this research, we formulated various initial solutions, and then, with the input of trusted and qualified mentors, proceeded to amend, revise, and ultimately converge onto an effective solution. Next, we completed a group charter, an itemized budget, a Gantt chart, a project proposal, and have already given two presentations to audiences composed of engineering professors and students. In order to complete the design stage, we are in the process of designing physical prototypes of the ComPAS and the AtLAs and have already ordered and received a majority of the sensors we will be using. This first stage is nearly complete, and once we are done we will be ready to integrate our prototypes into a cohesive product.\\

The next phase is integration. In the integration phase, we will be combining the working prototypes from the design phase into a complete product. This will involve finalizing the ComPAS and improving its robustness for transportation to the test location. The AtLAs will also need to be assembled and finalized, outfitted with buoyancy foam and protective boat fenders, and cleared for use in lakes and oceans. Also, we will need to reinforce both the AtLAs and the ComPAS and ensure that they are properly waterproofed and protected against the elements. Once the integration of our systems is complete, we can begin testing and assessing the performance of the systems.\\

The testing phase is the third and final stage. It includes a thorough investigation of our assembled systems and their ability to meet the established objectives. Unit tests have been created that will help streamline this process and prevent fixes from being introduced that adversely affect another piece of the system without being detected. As a final test, we will place the ComPAS on the shore of a lake or beach in Santa Cruz, deploy the AtLAs from the water, and then use the ComPAS and AtLAs to save a "drowning" volunteer. Once our system passes this minimum specification we will have completed our project.

\subsection*{Qualifications}

Darrel Deo is our team leader. He will also be working on integrating the sensors and building the tripod, which will be used to obtain the GPS coordinates of the drowning victim. Darrel is qualified for this position because of his work at MIT. While at MIT, Darrel worked with the Robotics, Vision, and Sensor Networks Group (RVSN) where he helped develop an assistive device for the visually impaired. This experience makes him well qualified for leading our team, and handling the sensor integration for the project.\\

Shehadeh Dajani is managing our budget. He will also be leading our Printed Circuit Board (PCB) design and layout, along with our vehicle design. Shehadeh is qualified for these positions because he has taken a course on PCB design and designed numerous PCBs both educationally and industrially. Shehadeh has also had previous experience with ROVs, at the University of Maryland. Finally, Shehadeh will be heading the human detection portion of the project by working with Darrel to design and modify existing sensors for aquatic human detection.\\

John Ash is managing our schedule for the project. He will also be working on the wireless protocols, water proofing, and PCB fabrication. John is qualified for these positions because of his previous experience with point-to-point communications. His experience with the autonomous solar boat on campus in the Autonomous Systems Lab (ASL) makes him an excellent consultant for protecting our hardware from water. Furthermore, John has hands-on experience fabricating PCBs.\\

David Goodman is our document administrator. He will also be developing the Finite State Machine (FSM) for our robot and the navigation system. David is more than qualified for this position, because he has five years of software engineering experience in industry. He is able to oversee design and testing of the software portion of our project. David has also worked in the ASL on the boat project, and with a graduate student on GPS-related research for the lab. David has the necessary experience to head the design and implementation of the software navigation system for our project.

\subsection*{Benefits}

This project will directly benefit the Santa Cruz community, as well as any coastal community around the world, by assisting a lifeguard to be more effective at saving people from drowning. Furthermore, the target budget for the Autonomous Lifeguard System, including research, is aimed at making this project a sensible and affordable end product that has the real opportunity to be used. The project is established in a way to allow for expansion from the original scope so that the AtLAs can be used for search and rescue or patrolling situations. Overall, we plan on using our engineering skills and talents to design a product with the sole purpose of saving lives and reducing the risk of drowning across the globe.

\end{document}
