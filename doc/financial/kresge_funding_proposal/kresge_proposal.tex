 \documentclass[11pt]{article}
\usepackage[pdftex]{graphicx}
\usepackage{siunitx}
\usepackage{verbatim}
\usepackage{float}
\usepackage{lastpage}
\usepackage{fancyhdr}
\usepackage[margin=1in, paperwidth=8.5in, paperheight=11in]{geometry}
%\usepackage{indentfirst}

% Header and Footer
\pagestyle{fancy}
%\fancyhead{} % clear all header fields
%\rhead{\vspace*{1em}\headone{David Goodman}\\
%	$118$ Hainline Road $\bullet$ Aptos, CA 95003\\
%	Phone: 714.363.1280\\
%	E-Mail: dagoodma@gmail.com}
%\cfoot{}

\renewcommand{\headrulewidth}{0pt}
%\renewcommand{\footrulewidth}{0.4pt}
%\cfoot{}
%\rfoot{Goodman \thepage}

\begin{document}
\begin{center} 
{
    {\Large \textbf{Autonomous Lifeguard}}\\
  	{\large Student Project Funds Proposal}
}
\end{center}

\section*{Abstract}
In the United States, there have been a reported 99 people that have drowned with a good number of them occurring while a lifeguard was on duty, in the past year alone. Lifeguards rescued around 63,000 individuals at risk of drowning. Last year alone, around 99 people have drowned with a good number of them occurring while a lifeguard was on duty. We are proposing an autonomous surface vehicle to aid and assist drowning victims during the critical minutes before help can reach them.\\


This project We are building an autonomous surface vessel that will aid a lifeguard on the beach in order to save someone from drowning. Our project allows a lifeguard to quickly target a drowning person with a magnifying scope that will obtain their GPS coordinates using calibrated encoders, and then communicate them to an autonomous boat located in the water beyond the shore break. The boat will, using its own on-board GPS, navigate itself to the person and allow them to hang on and stay afloat until the lifeguard arrives. This project aims to keep beaches safer by reducing the risk of drowning. This is an example of how I am most inclined to work on projects like this one that can preserve or improve the quality of life for all people.

This project is composed of two systems, a command center and autonomous vehicle, that will wirelessly communicate with each other. The command center will consist of a GPS-based scope mounted on a lifeguard post that will allow the user to obtain a coordinate location of a drowning victim. Once prompted, the command center communicates this waypoint to the autonomous vehicle stationed in the water. The vehicle will then navigate to the designated waypoint using an onboard GPS unit. Upon reaching this destination, the ASV will intelligently traverse the area until the drowning victim has grabbed onto the vehicle. The ASV will support the victim and allow them to rest while the lifeguard makes their way to the victim.

\section*{Narrative}

\subsection*{Background}

\subsection*{Objective}

\subsection*{Procedure}

In order to accomplish our objective of designing and implementing a device that can save someone from drowning, we are approaching this project in three phases: research and design, testing, and integration. Currently, we are in the research and design phase, which involves 


integration is an iterative process that may lead us back into the testing phase depending on the success that 

\subsection*{Qualifications}

\subsection*{Benefits}

\end{document}